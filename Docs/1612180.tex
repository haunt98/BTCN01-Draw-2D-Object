\documentclass[12pt]{article}
\usepackage[utf8]{vietnam}

% make indent in first paragraph
\usepackage{indentfirst}

\usepackage{url}

\title{BÁO CÁO BTCN-01: VẼ CÁC ĐỐI TƯỢNG ĐỒ HỌA 2D}
\author{Nguyễn Trần Hậu - MSSV 1612180}
\date{Tháng 10, 2018}

\begin{document}

\maketitle
\newpage
\tableofcontents
\newpage

\begin{abstract}
Trình bày cài đặt thuật toán vẽ các đối tượng 2D là
đường thẳng, đường tròn, elip, parabol, hyperbol
bằng các thuật toán là DDA, Bresenham, MidPoint.
So sánh và đánh giá từng thuật toán khi vẽ 1000, 5000, 10000 đối tượng.
Vẽ đường thẳng thêm thuật toán Xiaolin Wu.
\end{abstract}

\section{Thuật toán}

\subsection{Đường thẳng}

\subsubsection{DDA}
Vẽ một đường thẳng từ điểm \((x_1, y_1)\) đến \((x_2, y_2)\).
Tư tưởng chính là cho x chạy từ x1 đến x2,
sau đó tính lại y theo phương trình đường thẳng \(y=ax+b\).

Để dễ minh họa, xét trường hợp đơn giản \(x_1 < x_2\) và \(y_1 < y_2\).
Xét độ dốc \(m\) của phương trình đường thẳng:
\[m = dy/dx = (x_2 - x_1)/(y_2 - y_1)\]

Nếu \(m < 1\), \(x\) tăng nhanh hơn \(y\) và
ngược lại khi \(m > 1\) thì \(y\) tăng nhanh hơn \(x\).
Khi \(x\) tăng nhanh hơn \(y\), cho \(x\) tăng dần từ \(x_1\) đến \(x_2\)
rồi tính \(y\) theo \(x\), thì có thể vẽ được nhiều điểm hơn để tạo thành đường thẳng.
Khi \(y\) tăng nhanh hơn \(x\) thì làm ngược lại.

\subsubsection{Bresenham}
Sử dụng thuật toán DDA để vẽ đường thẳng, sẽ phải tính toán với số thực,
vì khi tính \(y\) theo \(x\) có thể là số thực.
Dùng thuật toán Bresenham thì chỉ tính toán với số nguyên.

Khi \(x_1 < x_2\) và \(y_1 < y_2\) và \(dy/dx < 1\),
thì \(x\) tăng nhanh hơn \(y\).
Do đó điểm tiếp theo của điểm \((x_i, y_i)\)
là \((x_i + 1, y_i)\) hoặc \((x_i + 1, y_i + 1)\).
Thực tế thì điểm tiếp theo phải là \((x_i + 1, y)\) nhưng y có thể là số thực.
Để xác định được cần chọn điểm nào, dùng:
\[d_1 = y - y_i\]
\[d_2 = y_i + 1 - y\]

Nếu \(d_1 < d_2\) chọn \(y_i\), ngược lại chọn \(y_i + 1\).
Đặt \(p_i = dx(d_1 - d_2)\), thì ta có:
\[p_{i+1} - p_i = 2dy - 2dx(y_{y+1} - y_i)\]

Với \(p_i < 0\)
\[p_{i+1} = p_i + 2dy\].

Với \(p_i >= 0\)
\[p_{i+1} = p_i + 2dy - 2dx\].

Bắt đầu vẽ từ điểm \((x_1, y_1)\), do đó
\[p_0 = 2dy - dx\]

Thuật toán bắt đầu từ việc xác định \(p_0\).
Sau đó trong lúc cho \(x\) chạy từ \(x_1\) đến \(x_2\),
kiểm tra \(p_{i+1}\) để chọn điểm vẽ tiếp theo là
\((x_i + 1, y_i)\) hay \((x_i + 1, y_i + 1)\).
Đối với trường hợp \(y\) tăng nhanh hơn \(x\), tráo thứ tự x và y.

\subsubsection{MidPoint}
Thuật toán MidPoint cũng chỉ tính toán với số nguyên.
Vẫn là giả sử \(x_1 < x_2\), \(y_1 < y_2\) và \(dy/dx < 1\) để x tăng nhanh hơn y.

Nếu như thuật toán Bresenham dùng \(d_1\) và \(d_2\)
để xác định điểm tiếp theo thì thuật toán MidPoint sử dụng điểm \((x + 1, y + 1/2)\)
bằng hàm \(F(x, y) = Ax + By + C\).
\(F(x, y) > 0 \) điểm nằm trên đường thẳng, ngược lại điểm nằm dưới đường thẳng.

Đặt \(p_i = 2F(x + 1, y+ 1/2)\), thì ta có
\[p_{i+1} = p_i + 2dy - 2dx(y_{i+1} - y_i)\]

Dễ thấy thuật toán MidPoint cho đường thẳng tính ra hệ số \(p_i\) và \(p_{i+1}\)
giống thuật toán Bresenham. Do đó đối với vẽ đường thẳng, thì MidPoint và Bresenham là một.

\subsubsection{Xiaolin Wu}
Thuật toán Bresenham vẽ đường thẳng nhanh nhưng không cho phép anti-aliasing.
Thuật toán Xiaolin Wu vẽ đường thẳng tương đối nhanh và cho phép anti-aliasing.

Giả sử \(x_1 < x_2\) và \(y_1 < y_2\). Xét trường hợp \(x\) tăng nhanh hơn \(y\).
Đặt \(gradient = dy/dx\). Khi \(x\) tăng dần từ \(x_1\) đến \(x_2\),
\(y\) lúc này là số thực tăng dần từ \(y_1\) với mỗi lần tăng \(gradient\).
Thuật toán vẽ 2 điểm \((x, int\_part(y))\) và \((x, int\_part(y) + 1)\) với màu sắc lần lượt là
\(1 - float\_part(y)\) và \(float\_part(y)\). Với \(int\_part\) là phần nguyên, \(float\_part\) là phần thực.
Khi \(y\) tăng nhanh hơn \(x\) thì hoán đổi vai trò của x và y.

\subsection{Đường tròn}

\subsubsection{DDA}
Đường tròn có thể chia ra thành \(8\) phần, trong đó mỗi phần đều đối xứng với nhau.
Do đó chỉ xét vẽ đường tròn từ điểm \((0, R)\) trong \(1/8\) đường tròn.
Nhận thấy lúc này \(x\) tăng nhanh hơn \(y\) giảm. Cho \(x\) tăng từ \(0\),
vẫn còn tăng khi điều kiện \(x < y\) vẫn còn thỏa, và tính lại \(y\) theo \(x\) bằng phương trình đường tròn.

\subsubsection{Bresenham}
Vẫn vẽ đường tròn bằng cách xác định điểm trong \(1/8\) đường tròn.
Đặt \(F(x, y) = x^2 + y^2 - R^2\). \(F > 0\) thì \((x, y)\) nằm ngoài đường tròn,
\(F < 0\) thì \((x, y)\) nằm trong đường tròn. Nếu đã vẽ được điểm \(F(x_i, y_i)\),
thì điểm tiếp theo cần chọn là \((x_i + 1, y_i)\) hoặc \((x_i + 1, y_i - 1)\).

Đặt \(p_i = F(x_i + 1, y_i) + F(x_i + 1, y_i - 1) \cite{bresenham_circle}\).

Với \(p_i < 0\) chọn \(y_{i+1} = y_i\)
\[p_{i+1} = p_i + 4x_{i+1} + 2\]

Với \(p_i >= 0\) chọn \(y_{i+1} = y_i - 1\)
\[p_{i+1} = p_i + 4x_{i+1} + 2 - 4y_{i+1}\]

Tại vị trí bắt đầu là điểm \((R, 0)\)
\[p_0 = 3 - 2R\]

Xác định \(p_0\) , cho \(x\) chạy từ \(0\) và tiếp tục tăng chừng nào điều kiện \(x < y\) còn thỏa mãn.
Dùng điều kiện so sánh \(p_i\) để chọn điểm tiếp theo là \((x_i + 1, y_i)\) hoặc \((x_i + 1, y_i - 1)\).

\subsubsection{MidPoint}
Vẫn vẽ đường tròn bằng cách xác định điểm trong \(1/8\) đường tròn.
Đặt \(F(x, y) = x^2 + y^2 - R^2\). Xét điểm \(F(x_i + 1, y_i - 1/2)\).

Đặt \(p_i = 4F(x_i + 1, y_i - 1/2)\).

Với \(p_i < 0\) chọn \(y_{i+1} = y_i\)
\[p_{i+1} = p_i + 8x_{i+1} + 4\]

Với \(p_i >= 0\) chọn \(y_{i+1} = y_i - 1\)
\[p_{i+1} = p_i + 8x_{i+1} + 4 - 8y_{i+1}\]

Tại vị trí bắt đầu là điểm \((R, 0)\)
\[p_0 = 5 - 4R\]

Cách chạy thuật toán MidPoint giống như thuật toán Bresenham
sau khi biết \(p_0\) và \(p_i\).

\subsection{Elip}

\subsubsection{DDA}
Hình Elip có thể chia ra làm \(4\) phần, mỗi phần đối xứng với nhau.
Do đó chỉ xét vẽ từng điểm trong \(1/4\) hình elip.
Trong đoạn này chia ra làm 2 đoạn nhỏ tùy vào \(dy/dx = -(xb^2)/(ya^2)\).
Đoạn đầu tiên \(x\) tăng nhanh hơn \(y\) giảm khi và chỉ khi \(dy/dx > -1\).
Đoạn thứ hai \(y\) tăng nhanh hơn \(x\) giảm khi và chỉ khi \(dy/dx < -1\).

Đoạn đầu tiên vẽ từ điểm \((0, b)\), sau mỗi lần tăng \(x\), tính lại \(y\) theo phương trình elip.
Vòng lặp chạy trong điều kiện \(dy/dx > -1\).
Đoạn thứ hai vẽ từ điểm \(a, 0\), sau mỗi lần tăng \(y\), tính lại \(x\).
Vòng lặp chạy trong điều kiện \(dy/dx < -1\).

\subsubsection{Bresenham}
Sử dụng cách chia elip như thuật toán DDA.
Đặt \(F(x, y) = x^2/a^2 + y^2/b^2 - 1\).

Đoạn đầu tiên vẽ từ điểm \((0, b)\), có
\[p_0 = 2b^2 - 2a^2 b + a^2\]
\[p_i = a^2 b^2 (F(x_i + 1, y_i) + F(x_i + 1, y_i - 1)) \cite{bresenham_ellipse}\]

Với \(p_i < 0\), \(y_{i+1} = y_i\)
\[p_{i+1} = p_i + 4b^2 x_{i+1} + 2b^2\]

Với \(p_i >= 0\), \(y_{i+1} = y_i - 1\)
\[p_{i+1} = p_i +  4b^2 x_{i+1} + 2b^2 - 4a^2 y_{i+1}\]

Đoạn thứ hai vẽ từ điểm \((a, 0)\), có
\[p_0 = 2a^2 - 2a b^2 + b^2\]
\[p_i = a^2 b^2 (F(x_i, y_i + 1) + F(x_i - 1, y_i + 1))\]

Với \(p_i < 0\), \(x_{i+1} = x_i\)
\[p_{i+1} = p_i + 4a^2 y_{x+1} + 2a^2\]

Với \(p_i < 0\), \(x_{i+1} = x_i - 1\)
\[p_{i+1} = p_i + 4a^2 y_{x+1} + 2a^2 - 4b^2 x_{i+1}\]

Thuật toán vẽ ở một đoạn chọn \(x\) hay \(y\) tăng liên tục,
sau đó dùng \(p_i\) để xác định điểm còn lại.

\subsubsection{MidPoint}
Sử dụng cách chia elip như thuật toán DDA.
Đặt \(F(x, y) = x^2/a^2 + y^2/b^2 - 1\).

Đoạn đầu tiên vẽ từ điểm \((0, b)\), có
\[p_0 = 4b^2 - 4a^2 b + a^2\]
\[p_i = 4a^2 b^2 F(x_k + 1, y_k - 1/2)\]

Với \(p_i < 0\), \(y_{i+1} = y_i\)
\[p_{i+1} = p_i + 8b^2 x_{i+1} + 4b^2\]

Với \(p_i >= 0\), \(y_{i+1} = y_i - 1\)
\[p_{i+1} = p_i + 8b^2 x_{i+1} + 4b^2 - 8a^2 y_{i+1}\]

Đoạn thứ hai vẽ từ điểm \((a, 0)\), có
\[p_0 = 4a^2 - 4a b^2 + b^2\]
\[p_i = 4a^2 b^2 F(x_i - 1/2, y_i + 1)\]

Với \(p_i < 0\), \(x_{i+1} = x_i\)
\[p_{i+1} = p_i + 8a^2 y_{k+1} + 4a^2\]

Với \(p_i >= 0\), \(x_{i+1} = x_i - 1\)
\[p_{i+1} = p_i + 8a^2 y_{i+1} + 4a^2 - 8b^2 x_{i+1}\]

Thuật toán vẽ ở một đoạn chọn \(x\) hay \(y\) tăng liên tục,
sau đó dùng \(p_i\) để xác định điểm còn lại.

\subsection{Parabol}

\subsubsection{DDA}
Xét phương trình Parabol \(y = ax^2/b\).
Parabol đối xứng có thể chia làm 2 phần đối xứng,
do đó chỉ cần vẽ một phần của parabol, phần còn lại lấy đối xứng qua.
Phương trình có độ dốc \(dy/dx = 2ax/b\).

Giả sử \(a/b > 0\), \(x\) tăng nhanh hơn \(y\) khi và chỉ khi \(dy/dx < 1\).
Ngược lại \(y\) tăng nhanh hơn \(x\) khi và chỉ khi \(dy/dx > 1\).

Với đoạn \(x\) tăng nhanh hơn \(y\), \(x\) chạy từ \((0, 0)\) đến chừng giới hạn của khoảng không gian 2d được vẽ.
Trong lúc đó \(y\) được tính lại theo \(x\). Khi \(y\) tăng nhanh hơn \(x\) thì làm ngược lại.

\subsubsection{Bresenham}
Như thuật toán DDA, chỉ cần vẽ \(1/2\) của parabol.
Đặt \(F(x, y) = y - ax^2/b\).

Trong đoạn \(x\) tăng nhanh hơn \(y\), điểm hiện tại đang vẽ là \((x_i, y_i)\)
\[d_1 = y - y_i\]
\[d_2 = y_i + 1 - y\]

Với \(d1 - d2 < 0\) thì điểm tiếp theo là \((x_i + 1, y_i)\),
ngược lại thì điểm tiếp theo là \((x_i + 1, y_i + 1)\).
Đặt \(p_i = b (d1 - d2)\), thì:
\[p_0 = 2a - b\]

Với \(p_i < 0\), thì \(y_{i+1} = y_i\)
\[p_{i+1} = p_i + 4a x_{i+1} + 2a\]

Với \(p_i > 0\), \(y_{i+1} = y_i + 1\)
\[p_{i+1} = p_i + 4a x_{k+1} + 2a - 2b\]

Trong đoạn \(y\) tăng nhanh hơn \(x\), điểm hiện tại đang vẽ là \((x_i, y_i)\)
\[d_1 = x^2 - x_i^2\]
\[d_2 = (x_i + 1)^2 - x^2\]

Đặt \(p_i = a (d1 - d2)\). \(p_0\) được tính từ điểm cuối cùng của
đoạn \(x\) tăng nhanh hơn \(y\).

Với \(p_i < 0\), thì \(x_{i+1} = x_i\)
\[p_{i+1} = p_i + 2b\]

Với \(p_i >= 0\), thì \(x_{i+1} = x_i + 1\)
\[p_(i+1) = p_i + 2b - 4a x_{i+1}\]

Thuật toán vẽ ở một đoạn chọn \(x\) hay \(y\) tăng liên tục,
sau đó dùng \(p_i\) để xác định điểm còn lại.

\subsubsection{MidPoint}
Như thuật toán DDA, chỉ cần vẽ \(1/2\) của parabol.
Đặt \(F(x, y) = y - ax^2/b\).

Trong đoạn \(x\) tăng nhanh hơn \(y\), xét điểm \((x + 1, y + 1/2)\).
Đặt \(p_i = 2b F(x + 1, y + 1/2)\), thì:
\[p_0 = b - 2a\]

Với \(p_i < 0\), thì \(y_{i+1} = y_i + 1\)
\[p_{i+1} = p_i + 2b - 4a x_{i+1} - 2a\]

Với \(p_i >= 0\), thì \(y_{i+1} = y_i\)

Trong đoạn \(y\) tăng nhanh hơn \(x\), xét điểm \((x + 1/2, y + 1)\).
Đặt \(p_i = 4b F(x + 1/2, y + 1)\).
\(p_0\) là giá trị tại điểm cuối cùng của đoạn \(x\) tăng nhanh hơn \(y\).

Với \(p_i < 0\), thì \(x_{i+1} = x_i\)
\[p_{i+1} = p_i + 4b\]

Với \(p_i >= 0\), thì \(x_{i+1} = x_i + 1\)
\[p_{i+1} = p_i + 4b - 8a x_{i+1} - 8a\]

Thuật toán vẽ ở một đoạn chọn \(x\) hay \(y\) tăng liên tục,
sau đó dùng \(p_i\) để xác định điểm còn lại.

\subsection{Hyperbol}

\subsubsection{DDA}
Phương trình hyperbol có dạng \(x^2/a^2 - y^2/b^2 - 1\).
Hyperbol chia làm 4 phần đối xứng với nhau,
do đó chỉ cần xác định điểm trên 1 phần, còn lại dùng phép đối xứng.
Phương trình có \(dy/dx = xb^2/ya^2\).
\(y\) tăng nhanh hơn \(x\) khi và chỉ khi \(dy/dx > 1\),
\(x\) tăng nhanh hơn \(y\) khi và chỉ khi \(dy/dx < 1\).

Trong đoạn \(y\) tăng nhanh hơn \(x\), cho \(x\) chạy từ \(a, 0\)
đến điểm giới hạn trong không gian vẽ 2d. Tính lại \(y\) theo \(x\).
Tiếp theo là đoạn \(x\) tăng nhanh hơn \(y\), thì tính lại \(y\) theo \(x\).

\subsubsection{Bresenham}
Như thuật toán DDA, chỉ cần xác định điểm cần vẽ trên \(1/4\) hyperbol.
Đặt \(F(x, y) = x^2/a^2 - y^2/b^2 - 1\).

Xét trong đoạn \(y\) tăng nhanh hơn \(x\).
\[d_1 = x^2 - x_i^2\]
\[d_2 = (x_i + 1)^2 - x^2\]

Đặt \(p_i = b^2 (d_1 - d_2) \cite{conic1} \), thì
\[p_0 = 2a^2 - 2ab^2 - b^2\]

Với \(p_i < -b^2/2\), thì \(x_{i+1} = x_i\)
\[p_{i+1} = p_i + 4a^2 y_{i+1} + 2a^2\]

Với \(p_i >= -b^2/2\), thì \(x_{i+1} = x_i + 1\)
\[p_{i+1} = p_i + 4a^2 y_{i+1} + 2a^2 - 4b^2 x_{i+1}\]

Xét trong đoạn tiếp theo \(x\) tăng nhanh hơn \(y\).
\[d_1 = y^2 - y_i^2\]
\[d_2 = (y_i + 1)^2 - y^2\]

Đặt \(p_i = a^2 (d_1 - d_2)\).
\(p_0\) là giá trị tại điểm cuối cùng của đoạn \(y\) tăng nhanh hơn \(x\).

Với \(p_i < -a^2/2\), thì \(y_{i+1} = y_i\)
\[p_{i+1} = p_i + 4b^2 x_{i+1} + 2b^2\]

Với \(p_i >= -a^2/2\), thì \(y_{i+1} = y_i + 1\)
\[p_{i+1} = p_i + 4b^2 x_{i+1} + 2b^2 - 4a^2 y_{i+1}\]

Thuật toán vẽ ở một đoạn chọn \(x\) hay \(y\) tăng liên tục,
sau đó dùng \(p_i\) để xác định điểm còn lại.

\subsubsection{MidPoint}
Như thuật toán DDA, chỉ cần xác định điểm cần vẽ trên \(1/4\) hyperbol.
Đặt \(F(x, y) = x^2/a^2 - y^2/b^2 - 1\).

Xét trong đoạn \(y\) tăng nhanh hơn \(x\).
Đặt \(p_i = 4a^2 b^2 F(x + 1/2, y + 1)\), thì
\[p_0 = 4ab^2 + b^2 - 4a\]

Với \(p_i < 0\), \(x_{i+1} = x_i + 1\)
\[p_{i+1} = p_i + 8b^2 x_{i+1} - 8a^2 y_{i+1} - 4a^2\]

Với \(p_i >= 0\), \(x_{i+1} = x_i\)
\[p_(i+1) = p_i - 8a^2 y_{i+1} - 4a^2\]

Xét đoạn tiếp theo, \(x\) tăng nhanh nhanh hơn \(y\).
Đặt \[p_i = 4a^2 b^2 F(x + 1, y + 1/2)\]

Với \(p_i < 0\), \(y_{i+1} = y_i\)
\[p_{i+1} = p_i + 8b^2 x_{i+1} + 4b^2\]

Với \(p_i >= 0\), \(y_{i+1} = y_i\)
\[p_(i+1) = p_i + 8b^2 * x_(k+1) + 4b^2 - 8a^2 y_(k+1)\]

Thuật toán vẽ ở một đoạn chọn \(x\) hay \(y\) tăng liên tục,
sau đó dùng \(p_i\) để xác định điểm còn lại.

\section{Đánh giá}



\begin{thebibliography}{1}
\bibitem{bresenham_circle}A Fast Bresenham Type Algorithm For Drawing Circles \\
\url{https://web.engr.oregonstate.edu/~sllu/bcircle.pdf}

\bibitem{bresenham_ellipse}A Fast Bresenham Type Algorithm For Drawing Ellipses \\
\url{https://dai.fmph.uniba.sk/upload/0/01/Ellipse.pdf}

\bibitem{midpoint_ellipse}Midpoint Ellipse Algorithm \\
\url{https://www.cpp.edu/~raheja/CS445/MEA.pdf}

\bibitem{conic1}Efficient integer algorithms for the generation of conic sections \\
\url{http://graphics.di.uoa.gr/Downloads/papers/journals/p7.pdf}

\bibitem{conic2}Generating Conic Sections Using an Efficient Algorithms \\
\url{https://www.iasj.net/iasj?func=fulltext&aId=38879}

\bibitem{wiki1}Midpoint circle algorithm \\
\url{https://en.wikipedia.org/wiki/Midpoint_circle_algorithm}

\bibitem{wiki2}Bresenham's line algorithm \\
\url{https://en.wikipedia.org/wiki/Bresenham%27s_line_algorithm}
\end{thebibliography}

\end{document}